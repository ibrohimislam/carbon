%--------------------------------------------------------------------%
%
% Berkas utama templat LaTeX.
%
% author Petra Barus, Peb Ruswono Aryan
%
%--------------------------------------------------------------------%
%
% Berkas ini berisi struktur utama dokumen LaTeX yang akan dibuat.
%
%--------------------------------------------------------------------%
\documentclass[12pt, a4paper, onecolumn, oneside, final, bahasa]{report}
\special{papersize=210mm,297mm}

% set margin
%\usepackage{showframe}
\usepackage[top=2.7cm,bottom=3cm,left=4cm,right=3cm]{geometry}

% set font
\usepackage{mathptmx}

% judul bahasa Indonesia
\usepackage[bahasa]{babel}

% satu setengah spasi
\renewcommand{\baselinestretch}{1.5}

\usepackage[utf8]{inputenc}

% citation style
\usepackage[backend=bibtex,style=authoryear]{biblatex}

\usepackage[activate={true,nocompatibility},final,tracking=true,kerning=true,spacing=true,factor=1100,stretch=10,shrink=10]{microtype}
\usepackage[T1]{fontenc}
\usepackage{graphicx}
\usepackage{titling}
\usepackage{blindtext}
\usepackage{sectsty}
\usepackage{chngcntr}
\usepackage[table,xcdraw]{xcolor}
\usepackage{textcomp}
\usepackage{float}
\usepackage{caption}
\usepackage{tabularx}
\usepackage{multirow}
\usepackage[toc,page,header]{appendix}

%remove spacing before chapter
\usepackage{titlesec}
\titleformat{\chapter}[display]
{\normalfont\large\bfseries\centering}{\chaptertitlename\ \thechapter}{8pt}{\centering\large}
\titlespacing*{\chapter}{0pt}{-20pt}{40pt}

%merapikan antara nomor list dan textnya
\usepackage{tocloft}
\setlength{\cfttabnumwidth}{1.5cm}
\setlength{\cftfignumwidth}{1.5cm}

\renewcommand{\cftpartpresnum}{\partname\hspace{10pt}}
\renewcommand{\cftchappresnum}{\chaptername\hspace{5pt}}

\renewcommand{\cftpartleader}{\cftdotfill{\cftdotsep}}
\renewcommand{\cftchapleader}{\cftdotfill{\cftdotsep}}
\renewcommand{\cftsecleader}{\cftdotfill{\cftdotsep}}

\renewcommand{\cftchapnumwidth}{4.2em}

\setlength{\cftbeforetoctitleskip}{-20pt}
\renewcommand{\cfttoctitlefont}{\hfill\large\bfseries}
\renewcommand{\cftaftertoctitle}{\hfill\hfill}

\setlength{\cftbeforeloftitleskip}{-20pt}
\renewcommand{\cftloftitlefont}{\hfill\hfill\large\bfseries}
\renewcommand{\cftafterloftitle}{\hfill\hfill}

\setlength{\cftbeforelottitleskip}{-20pt}
\renewcommand{\cftlottitlefont}{\hfill\hfill\large\bfseries}
\renewcommand{\cftafterlottitle}{\hfill\hfill}

% Daftar Lampiran %
\newcommand{\listappendixname}{Daftar Lampiran}
\newlistof{appendix}{app}{\listappendixname}
\setcounter{appdepth}{2}
\makeatletter
\AtBeginDocument{
	\newcommand{\@maketoatitle}{%
		\vspace*{\cftbeforeloftitleskip}
		\interlinepenalty\@M
		{\cftloftitlefont\listappendixname}{\cftafterloftitle}
		\cftmarklof
		\par\nobreak
		\vskip \cftafterloftitleskip
		\@afterheading
	}
	\renewcommand{\listofappendix}{
		\renewcommand{\cftchappresnum}{\appendixname\hspace{5pt}}
		\renewcommand{\cftchapnumwidth}{6em}
		\@cfttocstart
		\par
		\begingroup
			\@maketoatitle
			\parindent\z@ \parskip\z@
			\@starttoc{app}%
		\endgroup
	}
}
\makeatother
\newcommand{\chapterappendix}[1]{%
	\stepcounter{chapter}
	\chapter*{\appendixname\space\thechapter\linebreak#1}%
	\addcontentsline{app}{chapter}{\protect\numberline{\thechapter}#1}%
	\par
}
\newcommand{\sectionappendix}[1]{%
	\stepcounter{section}
	\section*{\thesection\quad#1}%
	\addcontentsline{app}{section}{\protect\numberline{\thesection}#1}%
}
\newcommand{\subsectionappendix}[1]{%
	\stepcounter{subsection}
	\subsection*{\thesubsection\quad#1}%
	\addcontentsline{app}{subsection}{\protect\numberline{\thesubsection}#1}%
}

\usepackage{listings}
\lstset{
	upquote=false,
	frame=tlrb,
	basicstyle=\ttfamily\footnotesize,
	stepnumber=1,
	showstringspaces=false,
	tabsize=4,
	breaklines=true,
	breakatwhitespace=false
}


\floatstyle{plaintop}
\restylefloat{table}

\counterwithin{figure}{section}
\counterwithin{table}{section}

\captionsetup[table]{skip=10pt}
\captionsetup[figure]{name={Gambar.},labelsep=period}

\makeatletter
\def\@kampus{PROGRAM STUDI TEKNIK INFORMATIKA\\
	SEKOLAH TEKNIK ELEKTRO DAN INFORMATIKA\\
	Institut Teknologi Bandung}

\newcommand{\MONTH}{%
	\ifcase\the\month
	\or Januari% 1
	\or Februari% 2
	\or Maret% 3
	\or April% 4
	\or Mei% 5
	\or Juni% 6
	\or Juli% 7
	\or Agustus% 8
	\or September% 9
	\or Oktober% 10
	\or November% 11
	\or Desember% 12
	\fi}

%\def\@maketitle{   % custom maketitle
\newcommand{\judul}{   % custom maketitle

\begin{center}
\smallskip
\large \bfseries \MakeUppercase{\thetitle}
\vfill
\large Laporan Tugas Akhir\\
\vfill
Disusun sebagai syarat kelulusan tingkat sarjana\\
\vfill
oleh\\
\large \theauthor
\vfill
\begin{figure}[h]
\centering
  	\includegraphics[width=0.15\textwidth]{resources/cover-ganesha.jpg}
\end{figure}
\vfill
\large
\@kampus \\
\large
\MONTH\space\the\year

\end{center}
}
\makeatother

\sectionfont{\normalsize}
\subsectionfont{\normalsize}

\renewcommand{\thechapter}{\Roman{chapter}}

\bibliography{references}

\begin{document}

\title{Implementasi Deep Packet Inspection pada Iptables sebagai Penangkal Infeksi Malware Melalui Jaringan}
\date{}
\author{IBROHIM KHOLILUL ISLAM\\
NIM : 13513090}

\input{chapters/cover}
\begin{center}
	\large
	\textbf{\thetitle}
	\small
\end{center}

\begin{center}
	
	\vfill
	\textbf{Draf Laporan Tugas Akhir II}\\
	\vfill
	oleh\\
	\theauthor
	\vfill
	Program Studi: Teknik Informatika \\
	Sekolah Teknik Elektro dan Informatika \\
	Institut Teknologi Bandung
	\vfill
	Bandung, 7 Desember 2018 \\
	Mengetahui,\\
	Pembimbing \\
	\vspace{60pt}
	Yudistira Dwi Wardhana Asnar ST, Ph.D.\\
	NIP. 19800827 201504 1 002
\end{center}

\pagenumbering{roman}
\setcounter{page}{0}

\pagestyle{plain}

\clearpage
\chapter*{ABSTRAK}
\addcontentsline{toc}{chapter}{Abstrak}

Pada tugas akhir ini, dikembangkan sebuah \textit{inline network-based malware detection} untuk melakukan deteksi malware. Hal ini dilakukan karena firewall saat ini masih belum dapat melakukan deteksi infeksi malware. Malware yang dipakai pada pengembangan ini adalah malware WannaCry yang menginfeksi lebih dari 75.000 host pada 2017. \textit{Inline network-based malware detection} diterapkan pada sebuah firewall sehingga dapat melakukan penagkalan infeksi.

Pengembangan dilakukan dengan menggunakan teknik \textit{dynamic signature-based}. Teknik ini secara teori dapat meminimalisir \textit{false-negative} dibandingkan dengan teknik \textit{anomaly-based}. Signature yang dibentuk dari hasil inspeksi oleh DPI yang dikembangkan untuk protokol SMB. Signature dalam bentuk \textit{state machine} dari hasil inspeksi oleh SMB digunakan untuk menentukan apakah sebuah paket berbahaya atau tidak.

Pengujian dilakukan dengan melakukan percobaan. Percobaan dilakukan dengan menerapkan transparent-firewall dalam sebuah subnet. Hasil percobaan menunjukan implementasi \textit{inline network-based malware detection} tersebut dapat melakukan penangkalan pada 83 percobaan yang telah dilakukan. Dalam percobaan tersebut belum menunjukan ditemukannya \textit{false-negative}. Namun pada perancangan pengujian tidak dilakukan untuk mendeteksi \textit{false-positive}.

\vspace{17px} \noindent Kata kunci: \textit{network-based malware detection}, firewall, wannacry

\clearpage
\chapter*{Kata Pengantar}
\addcontentsline{toc}{chapter}{Kata Pengantar}

Puji dan syukur penulis panjatkan ke hadirat Tuhan Yang Maha Esa. Berkat rahmat-Nya, penulis mampu menyelesaikan Tugas Akhir yang berjudul ‘\thetitle’. Penulis juga mengucapkan terima kasih kepada:
\begin{enumerate}
	\item Dosen pembimbing, Bapak Yudistira Dwi Wardhana Asnar yang telah meluangkan waktu untuk membimbing penulis dengan memberikan ide dan saran.
	\item Staf Prodi Teknik Informatika ITB, terutama para dosen yang telah membagikan ilmu dan tata usaha yang membantu pengurusan administrasi tugas akhir. 
	\item Orang tua dan anggota keluarga lainnya yang selalu mendoakan dan memotivasi penulis.
	\item Teman-teman yang selalu mengingatkan penulis untuk melanjutkan pengerjaan tugas akhir serta memberikan semangat.
\end{enumerate}

Penulis menyadari bahwa tugas akhir ini masih memiliki banyak kekurangan sehingga penulis terbuka untuk kritik dan saran dari pembaca. Semoga tugas akhir ini dapat bermanfaat bagi pembaca.

\null\hfill Bandung, \today\\
\vspace{15pt}\\
\null\hfill Penulis
\clearpage

\tableofcontents
\clearpage
\listofappendix
\clearpage
\listoffigures
\clearpage
\listoftables
\clearpage

\pagenumbering{arabic}
\setcounter{page}{1}

\chapter{Pendahuluan}

\section{Latar Belakang}

\textit{First-generation} firewall sudah tidak dapat lagi memenuhi kebutuhan keamanan. Firewall ini tidak dapat melakukan pencegahan intrusi pada level protokol yang lebih tinggi. Akibatnya, serangan-serangan dapat dilakukan meskipun suatu enterprise telah mengunakan firewall.

Kekurangan yang dimiliki \textit{first-generation} firewall, menjadikan \textit{enterprise} harus penggunakan sistem lain untuk dapat mendeteksi serangan. Sistem lain seperti \textit{Intrusion Detection System} (IDS) dan \textit{Intrusion Prevention System} (IPS). Hal itu menimbulkan biaya lebih yang harus dikeluarkan oleh \textit{enterprise}.

\textit{Next-generation} firewall menurut (Pescatore, 2009) menjadi solusi untuk menyelesaikan masalah ini. Next-generation firewall merupakan firewall yang memiliki fitur first-generation firewall, namun memiliki fitur-fitur IDS maupun IPS yang terintegrasi, dengan menitikberatkan pada deep packet inspection (DPI).

Menurut sebuah survey dari Alcatel-Lucent strategic white paper (The Case for
Network-based Malware Detection, 2014), solusi pengamanan client-based kurang
efektif untuk mencegah malware. Hampir 81\% terinfeksi malware, meskipun telah
dipasangi oleh anti-virus. Sehingga diperlukan sebuah solusi pengamanan yang
tidak dapat di-disable dari sistem. Salah satu solusinya adalah dengan
menggunakan network-based malware detection.

Malware detection menjadi fitur penting yang diperlukan oleh sebuah sistem pengamanan. Namun, saat ini implementasi open-source firewall, yakni pfSense masih memberikan fitur native untuk melakukan deteksi malware pada \textit{layer} aplikasi.

\section{Rumusan Masalah}
Implementasi open-souce firewall (pfSense, 2017) dan (OPNsense, 2017) saat ini
belum melakukan inspeksi tehadap application layer protocol secara generic.
Padahal intrusi seperti malware pada umumnya sulit dideteksi pada transport-layer,
sehingga masih sulit dilakukan.

\section{Tujuan}
Tujuan dari tugas akhir ini adalah:
\begin{enumerate}
	\item Menentukan teknik yang extensible untuk melakukan pendeteksian malware pada network untuk diimplementasikan pada firewall open-source.
	\item Membangun sistem pendeteksi malware pada implementasi firewall open-source.
\end{enumerate}

\section{Batasan Masalah}
Implementasi Network-Based Malware Detection pada firewall open-source
difokuskan pada hal-hal sebagai berikut:
\begin{enumerate}
	\item Implementasi tidak difokuskan untuk meningkatkan performa firewall
	setelah Network-Based Malware Detection diimplementasi.
	\item Implementasi menggunakan komponen-komponen yang sudah ada dari perangkat lunak open-source dengan kapabilitas deep packet inspection.
	\item Implementasi akan dibatasi untuk melakukan deteksi pada \textit{malware} WannaCry.
\end{enumerate}

\section{Metodologi}
Metodologi yang digunakan pada pengerjaan makalah ini, antara lain:
\begin{enumerate}
	\item Studi literatur. Pada studi literatur, dilakukan pencarian referensi mengenai
	definisi-definisi pada domain next-generation firewall. Referensi juga
	digunakan untuk mendapatkan \textit{state of the art} dari domain ini.
	\item Analisis masalah. Dalam tahapan ini, dilakukan analisis metode apa saja
	yang dapat dilakukan untuk mendeteksi malware dalam seuah network, dan sampai sejauh mana implementasi next-generation firewall dapat
	menangani masalah ini.
	\item Perancangan solusi. Hasil analisis masalah kemudian dilakukan pemilihan
	solusi yang tepat dan cukup generic dalam mendeteksi malware.
	\item Implementasi. Pada tahap ini hasil rancangan solusi kemudian
	diimplementasikan pada \textit{open-source next-generation firewall} yang sudah
	ada.
	\item Pengujian dan analisis hasil. Pada tahap ini hasil implementasi dilakukan
	pengujian dengan menggunakan beberapa kasus uji.
\end{enumerate}


\chapter{Tinjauan Pustaka}

\section{Network Border Security}
Pada (Strebe, 2004) border-security secara teori harus memiliki measures sebagai
berikut:
\begin{enumerate}
	\item \textit{Control every crossing}
	
	Border security harus melakukan pengecekan untuk setiap lalu lintas data antara internal network dan external network. Sebuah koneksi antara internal network dan external network yang tidak dilakukan pengecekan
	dapat menjadi celah untuk terjadinya serangan.
	
	\item \textit{Apply the same policy universally}
	
	Sebuah control untuk sebuah lalu lintas data tertentu harus dilakukan sama untuk seluruh hubungan yang terjadi antara internal network dan external network. Hal ini membutuhkan penerapan menyeluruh, karena efek dari penerapan ini akan bergantung pada penerapan yang terlemah.
	
	\item \textit{Deny by default}
	
	Seluruh keterhubungan hanya akan memperbolehkan lalu lintas data yang ada pada whitelist. Penerapan ini perlu dilakukan untuk lalu lintas ke luar	maupun ke dalam firewall.
	
	\item \textit{Hide as much as information as possible}
	
	Penyembunyian data interior dari sebuah network perlu dilakukan. Hal ini digunakan untuk mencegah penyerang mendapatan informasi internal network.
	
\end{enumerate}

\section{Firewall}
Firewall merupakan hardware, software atau kombinasi keduanya yang digunakan untuk melakukan monitoring dan filter terhadap lalu lintas data yang masuk atau keluar dari sebuah jaringan yang berusaha dilindungi. (Kizza, 2005)

Menurut (Stallings, 2012) \textit{design goal firewall} sebagai berikut:
\begin{itemize}
	\item Seluruh lalu lintas data dari dalam ke luar ataupun sebaliknya, harus melalui
	firewall.
	\item Hanya lalu lintas data yang terotorisasi yang dapat melalui firewall.
	\item Firewall merupakan system yang kebal terhaadap penetration.
\end{itemize}

Kekuatan dan kelemahan dari firewall menurut (Peterson, 2012):
\begin{itemize}
	\item Firewall dapat dideploy unilaterally
	\item Firewall tidak dapat membatasi akses antara host yang berada dalam internal-network
	\item Jika pihak diberikan akses ke internal-network, maka pihak tersebut
	menjadi security vulnerability.
	\item Bug pada firewall yang dapat diakses dari internal-network dapat menjadi masalah serius.
\end{itemize}

Firewall saat ini yang pada umumnya memiliki tipe sebagai berikut:

\begin{enumerate}
	\item \textit{Packet filtering firewall}
	\textit{Packet filtering firewall}
	
	merupakan firewall yang menggunakan informasi dari protokol IP untuk menentukan apakah dilakukan teruskan atau buang untuk lalu lintas data masuk maupun keluar.
	
	\item \textit{Application proxy firewall}
	
	Application proxy firewall atau application gateway merupakan firewall yang digunakan pada application layer untuk sebuah protocol tertentu.
	
\end{enumerate}

\section{Arsitektur Firewall}
Firewall pada implementasinya dapat ditempatkan dengan beberapa arsitektur menurut (\cite{zwicky2000building}).

\subsection{Arsitektur \textit{Dual-Homed Host}}

Arsitektur \textit{dual-homed host} merupakan dibangun dengan menggunakan komputer \textit{dual-homed host}, yakni sebuah komputer yang terhubung dengan dua atau lebih jaringan. Komputer ini dapat bekerja sebagai router antara kedua jaringan yang terhubung ke komputer tersebut. Namun, untuk mengimplemntasi firewall dengan arsitektur \textit{dual-homed host} fungsi routing ini tidak difungsikan. Sehingga tidak ada data yang dapat dikirimkan langsung antar kedua jaringan. Jadi untuk setiap paket yang akan dikirimkan dari jaringan internal ke jaringan luar harus melalui \textit{dual-homed host}, dan dari jaringan luar ke jaringan dalam juga harus melalui \textit{dual-homed host}. Sehingga susunan komponen pada jaringan tersebut seperti pada gambar \ref{fig:dual_homed}

\begin{figure}[H]
	\centering
	\includegraphics[width=300px]{resources/dual_homed.png}
	\caption{Arsitektur \textit{Dual-Homed}}
	\label{fig:dual_homed}
\end{figure}

\subsection{Arsitektur \textit{Screened Host}}

Berbeda dengan arsitektur \textit{dual-homed host} yang memberikan service dari host yang terhubung ke beberapa jaringan dengan fungsi tidak mengaktifkan fungsi routing, arsitektur \textit{screened host} memberikan layanan dari host yang hanya terhubung ke internal network dan menggunakan router terpisah seperti pada  gambar \ref{fig:screened_host}. Pada arsitektur ini, keamanan dijaga oleh packet filtering, dengan melakukan konfigurasi berikut:

\begin{enumerate}
\item Memperbolehkan \textit{internal host} untuk dapat mengakses beberapa service langsung tanpa melalui proxy.
\item Melarang semua paket dari internal host. (Untuk memaksa host itu menggunakan proxy).
\end{enumerate}

Namun pada pada arsitektur ini, jaringan internal sangat mudah diserang dari host \textit{bastion} yang berada pada internal network. Sehingga bastion host menjadi sasaran yang paling diinginkan oleh penyerang. Karena tidak ada pertahanan lagi diantara internal host dan host lain yang berada pada jaringan internal.

\begin{figure}[H]
	\centering
	\includegraphics[width=300px]{resources/screened_host.png}
	\caption{Arsitektur \textit{Screened Host}}
	\label{fig:screened_host}
\end{figure}

\subsection{Arsitektur \textit{Screened Subnet}}

Arsitektur \textit{screened subnet} memiliki layer tambahan dibandingkan dengan arsitektur \textit{screened host} dengan memisahkan internal network lebih jauh dari Internet.

Arsitektur ini ditujukan agar host \textit{bastion}, yakni host yang dieskpos ke Internet, merupakan host yang paling rentan untuk diserang. Meskipun host sudah dilakukan usaha untuk melindungi host tersebut, namun host tersebut menjadi titik paling jelas untuk diserang.

\begin{figure}[H]
	\centering
	\includegraphics[width=300px]{resources/screened_subnet.png}
	\caption{Arsitektur \textit{Screened Subnet}}
	\label{fig:screened_subnet}
\end{figure}

\section{Next Generation Firewall}
Pada masa ini, firewall terbagi menjadi 2 yaitu tradisional firewall dan new generation firewall. Hal ini terjadi karena tradisional firewall tidak lagi mumpuni untuk menahan serangan yang ada di dunia internet ini. 
Tradisional firewall adalah firewall yang bekerja di \textit{network layer}(\cite{nicoll2004challenges}), menggunakan port dan protokol IP untuk mengontrol dan mencegah serangan dari jaringan.(\cite{zhong2012design}) Skema dari tradisional firewall ini dapat dilihat pada gambar berikut.
\begin{figure}[H]
	\centering
	\includegraphics[width=0.8\textwidth]{resources/tradisional_firewall.png}
	\caption{Skema tradisional firewall(\cite{zhong2012design})}
	\label{fig:tradisional_firewall}
\end{figure}

Firewall ini hanya mengecek header paket sesuai dengan portnya, sehingga tidak dapat mengontrol aplikasi. Tradisional firewall yang memakai \textit{deep packet inspection} (DPI) untuk menambah keamanan juga tidak terlalu berhasil, karena memunculkan limitasi dan masalah baru. Menurut (\cite{miller2011next}), masalah tersebut adalah
\begin{itemize}
	\item Aplikasi yang tidak seharusnya berada di jaringan diperbolehkan masuk ke jaringan.
	\item Tidak semua paket yang harus diperiksa terperiksa
	\item Policy management menjadi rumit dan berbelit
	\item Performansi yang tidak memadai 
\end{itemize}

Sementara itu, Next Generation Firewall (NGFW) merupakan pengembangan dari first-generation firewall yang memiliki Deep Packet Inspection (DPI). Secara fungsionalitas NGFW merupakan gabungan dari IPS dan first-generation firewall. NGFW dapat dipandang sebagai IPS karena NGFW memiliki awareness terhadap application level payload. Hasil pengecekan kemudian digunakan untuk memutuskan apakah paket di-forward atau di-drop. (Pescatore, 2009). New Generation Firewall adalah firewall yang berjalan diatas layer aplikasi, untuk mendeteksi apakah paket tersebut sesuai dengan user`s rule atau tidak(\cite{zhong2012design}). Skema dari new generation firewall adalah sebagai berikut.
\begin{figure}[H]
	\centering
	\includegraphics[width=\textwidth]{resources/NGFW.png}
	\caption{Skema new generation firewall(\cite{zhong2012design})}
	\label{fig:new_generation_firewall}
\end{figure}

Fungsi dan kemampuan utama yang dibutuhkan oleh new generation firewall adalah:
\begin{itemize}
	\item Identifikasi aplikasi (port, protocol, evasive techniques, atau SSL encryption) sebelum melakukan hal apapun
	\item Menyediakan policy-based control yang lebih jelas dan granular
	\item Secara akurat mengidentifikasi pengguna dan menggunakan informasi tersebut sebagai atribut dari policy control
	\item Menyediakan proteksi secara real-time terhadap ancaman dari jaringan, termasuk yang beroperasi pada layer aplikasi.
	\item Terintegrasi untuk meningkatkan kapasitas pencegahan ancaman
\end{itemize}

\subsection{Teknik Identifikasi aplikasi}
Teknik Identifikasi aplikasi yang digunakan oleh new generation firewall adalah
\begin{itemize}
	\item Deteksi dan dekripsi aplikasi protokol\\
	Untuk mendeteksi protokol aplikasi sehingga dapat dianalisis lebih lanjut
	\item Decode aplikasi protokol\\
	Untuk mendeteksi apakah ada aplikasi lain yang berjalan melalui protokol tersebut (tunnel), seperti Yahoo! Instant Messenger yang mungkin berada di dalam protokol HTTP
	\item Application signatures\\
	Untuk mengecek apakah aplikasi tersebut menggunakan port dan protokol yang sesuai dengan fungsinya.
	\item Heuristics
\end{itemize}

\subsection{User identification}
Teknologi ini mengidentifikasi user sehingga dapat digunakan untuk:
\begin{itemize}
	\item Mendapatkan visibilitas tentang siapa yang bertanggung jawab untuk semua aplikasi, konten, dan ancaman lalu lintas data pada jaringan tersebut
	\item Mengizinkan penggunaan identitas sebagai variabel dalam access control policies.
	\item Memfasilitasi troubleshooting/incident response
\end{itemize}

\subsection{Content identification}
Teknologi ini membuat next generation firewall dapat mencegah ancaman secara real-time, mengontrol web surfing activities, dan memfilter file atau data. Komponen dari teknologi ini adalah:
\begin{itemize}
	\item Pencegahan ancaman\\
	Komponen ini berfungsi untuk mencegah spyware, virus, dan ancaman lainnya dari jaringan. Komponen ini dibantu oleh application decoder, stream-based virus detection, spyware scanning, uniform threat signature format, dan IPS.
	\item URL filtering\\
	Komponen ini memfilter konten melalui URL.
	\item Filter file dan data\\
	Komponen ini menggunakan kelebihan dari in-depth application inspection untuk mengurangi pengiriman file dan data yang tidak terotorisasi. 
\end{itemize}

\subsection{Perbedaan performansi antara tradisional firewall dan new generation firewall}
Pada tradisional firewall, fungsi-fungsi keamanan dilakukan secara terpisah satu sama lain, seperti yang digambarkan pada gambar \ref{fig:architecture_tradisional_firewall}. Hal ini menyebabkan penggunaan system resource yang berlebihan dan tidak efisien.
\begin{figure}[H]
	\centering
	\includegraphics[width=\textwidth]{resources/architecture_tradisional_firewall.png}
	\caption{arsitektur proses tradisional firewall(\cite{miller2011next})}
	\label{fig:architecture_tradisional_firewall}
\end{figure}

Sebaliknya, new generation firewall menggunakan single-pass architecture untuk mengeliminasi pengecekan paket secara repetitif, mengurangi beban pada hardware dan meminimalkan latency, seperti yang dapat dilihat pada gambar berikut.
\begin{figure}[H]
	\centering
	\includegraphics[width=0.7\textwidth]{resources/architecture_NGFW.png}
	\caption{arsitektur proses new generation firewall(\cite{miller2011next})}
	\label{fig:architecture_NGFW}
\end{figure}

\section{Malware}
\textit{Malicious software} menurut (\cite{idika2007survey}) memiliki banyak baentuk salah satunya adalah Malicious Code (MC). Menurut (\cite{attackingmalcode}), malicious code merupakan kode yang ditambahkan, diubah, atau dihilangkan dari sistem software untuk membahayakan atau mengubah fungsi yang diharapkan dapat dilakukan oleh sistem.

Virus merupakan program komputer yang mereplikasi dengan menyisipkan dirinya ke program lain. Program yang disisipi oleh virus menjadi terinfeksi disebut inang. Hal ini yang membedakan virus dengan malware lain, untuk melakukan fungsinya virus memerlukan inang (\cite{attackingmalcode}).

Worm merupakan program komputer yang mereplikasi dirinya dengan menjalankan kode worm yang manjadi sebuah program tersendiri. Bagian yang membedakan worm dan virus, worm tidak memerlukan inang untuk melakukan fungsinya. Selain itu, virus dan worm juga berbeda pada cara penyebarannya. Pada umumnya, virus berusaha untuk menyebar melalui file atau program pada satu komputer. Sedangkan worm berusaha untuk menginfeksi sebanyak mungkin komputer melalui jaringan (\cite{attackingmalcode}).

Trojan horse merupakan malicious code yang tambahkan oleh desainer dalam sebuah aplikasi atau sistem. Aplikasi tersebut menjalankan fungsinya, namun melakukan aktifitas malicious seperti merekam kegiatan pengguna dan mengirimkannya ke pembuatnya. Trojan horse pada umumnya berkaitan dengan mengakses dan mengirimkan informasi tanpa otorisasi dari pengguna. Trojan horse dapat dikategorikan sebagai spyware. 


\section{Malware Detection}

Malware merupakan singkatan dari malicious software. Malware pada umumnya didesain untuk dapat menyebarkan dirinya untuk dapat berkembang. Pada (\cite{idika2007survey}) teknik untuk mendeteksi malware dibedakan menjadi dua, yakni: signature-
based dan anomaly-based. Terdapat bentuk khusus anomaly-based yakni specification-based. Setiap teknik memiliki jenis static, dynamic, dan hybrid.

\subsection{Anomaly-based Detection}

Pada (\cite{idika2007survey}) dijelaskan, anomaly-based detection memiliki dua fasa, yakni fasa training, dan fasa deteksi. Pendeteksian jenis ini memiliki kelebihan, yakni dapat mendeteksi serangan yang sebelumnya belum dikenali. Namun, deteksi jenis ini memiliki tingkat kesalahan pendeteksian yang tinggi. Anomaly based-detection melakukan pendeteksian dengan cara membuat pendekatan perilaku yang valid dilakukan oleh sebuah sistem.

\begin{figure}[H]
	\centering
	\includegraphics[width=220px]{resources/anomaly_illustration.png}
	\caption{Ilustrasi karakterisasi perilaku pada anomaly-based detection}
	\label{fig:anomaly_illust}
\end{figure}

Gambar \ref{fig:anomaly_illust} menggambarkan bagaimana anomaly-based detection mengkarakterisasi perilaku sistem. V merupakan himpunan perilaku yang tidak bertentangan dengan requirement. V’ merupakan himpunan perilaku yang yang tidak valid. Vapprox merupakan hasil pendekatan yang dilakukan oleh anomaly-based detection.

\subsection{Signature-based Detection}
Signature-based detection dalam (\cite{idika2007survey}) merupakan teknik yang menggunakan malicious-model untuk mendeteksi malware. Kumpulan dari malicious-model (signature) menjadi knowledge base dari system pendeteksi jenis ini. Sehingga pada Gambar \ref{fig:signature_illust}, diilustrasikan bahwa S (kumpulan signature) merupakan subset dari U yang merupakan seluruh signature dari perilaku malicious. Karena keterbatasan media penyimpanan, S akan sangat kecil jika dibandingkan dengan U yang sangat besar.

\begin{figure}[H]
	\centering
	\includegraphics[width=190px]{resources/signature_illustration.png}
	\caption{Ilustrasi himpunan signature terhadap seluruh malicious signature}
	\label{fig:signature_illust}
\end{figure}

\section{Iptables}

Menurut (\cite{purdy2004linux}) Netfilter adalah bagian dari kernel Linux yang berfungsi untuk memproses paket dari jaringan. Iptables adalah perintah untuk mengatur \textit{Netfilter} yang dapat digunakan pada \textit{user-space}. Arsitektur dari iptables dikelompokkan berdasarkan fungsinya, yaitu
\begin{itemize}
	\item Filter\\
	Digunakan untuk mengatur keluar masuknya paket.
	\item \textit{Network address translation} (NAT)\\
	Digunakan dengan \textit{connection tracking} untuk melakukan Network Address Translation (NAT).
	\item Packet mangling\\
	Digunakan untuk memanipulasi paket.	
\end{itemize} 

Fungsi-fungsi tersebut memiliki \textit{chain} (urutan) pemrosesannya masing-masing. Aturan tersebut berisikan syarat (\textit{matches}) dan target. Syarat dari aturan tersebut akan menentukan paket mana saja yang akan terkena aturan tersebut, sementara target akan menentukan apa yang akan dilakukan oleh paket yang memenuhi syarat tersebut. Apabila tidak ada syarat (\textit{match criteria}) maka semua paket dianggap memenuhi syarat. Sebaliknya, apabila tidak ada target, maka paket tidak akan diproses. Syarat yang dapat digunakan antara lain adalah IP (\textit{Internet Protocol}) dan \textit{MAC addresses}.
Netfilter memiliki beberapa \textit{built-in target}, yaitu:
\begin{itemize}
	\item \textit{ACCEPT}\\
	Mengijinkan paket untuk menuju proses selanjutnya.
	\item \textit{DROP}\\
	Menghentikan proses paket sepenuhnya.
	\item \textit{QUEUE}\\
	Mengirimkan paket ke userspace.
	\item \textit{RETURN}\\
	Menghentikan proses pada user-defined chain dan melanjutkan proses ke chain sebelum user-defined chain dipanggil.
\end{itemize}
\subsection{Hooks point}
\textit{Netfilter} memiliki 5 poin didalam alur pemrosesan paket, yaitu :
\begin{itemize}
	\item \textit{PREROUTING}\\
	Poin yang akan memproses paket yang baru datang dari \textit{network interface} (setelah melalui proses pengecekan \textit{checksum}).
	\item \textit{INPUT}\\
	Poin yang akan memproses paket yang akan dilanjutkan ke proses lokal.
	\item \textit{FORWARD}\\
	Poin yang akan memproses paket yang keluar masuk melalui \textit{gateway} komputer.
	\item \textit{POSTROUTING}\\
	Poin yang akan memproses paket yang akan meninggalkan \textit{network interface}.
	\item \textit{OUTPUT}\\
	Poin yang akan memproses paket yang baru saja melewati proses lokal.
\end{itemize}
aturan dasar (\textit{built-in chain}), dan dapat ditambahkan aturan lain yang diinginkan.

\subsection{Alur pemrosesan paket}
Fungsi-fungsi dari arsitektur iptables tersebut disebut juga \textit{tables}, dan fungsi-fungsi tersebut memiliki alur pemrosesan dasar sebagai berikut.\\
\begin{figure}[H]
	\centering
	\includegraphics[width=\textwidth]{resources/nat_table.png}
	\caption{Alur pemrosesan paket dan \textit{hook} yang digunakan pada tabel \textit{NAT}.}
	\label{fig:packetflow_NAT}
\end{figure}

\begin{figure}[H]
	\centering
	\includegraphics[width=\textwidth]{resources/filter_table.png}
	\caption{Alur pemrosesan paket dan \textit{hook} yang digunakan pada tabel \textit{filter}.}
	\label{fig:packetflow_filter}
\end{figure}

\begin{figure}[H]
	\centering
	\includegraphics[width=\textwidth]{resources/mangling_table.png}
	\caption{Alur pemrosesan paket dan \textit{hook} yang digunakan pada tabel \textit{mangling}.}
	\label{fig:packetflow_mangling}
\end{figure}

Paket yang melewati \textit{chain} akan terkena aturan sesuai dengan urutannya. Apabila paket tersebut tidak sesuai kriteria, maka paket akan bergerak ke aturan selanjutnya. Jika paket tersebut tidak memenuhi kriteria sampai aturan yang paling akhir, maka paket tersebut akan diperlakukan sesuai dengan \textit{chain\textquotesingle s policy}. 
Berdasarkan Gambar , urutan alur paket yang melewati \textit{tables} dan \textit{chains} dapat dilihat pada tabel-tabel berikut.

\begin{table}[H]
	\caption{Alur paket yang melewati network interface ke network interface lainnya (forwarding)}
	\label{table:network_to_network}
	\centering
	\begin{tabular}{ll}
		\hline
		\rowcolor[HTML]{C0C0C0} 
		table  & chain       \\ \hline
		mangle & PREROUTING  \\
		nat    & PREROUTING  \\
		mangle & FORWARD     \\
		filter & FORWARD     \\
		mangle & POSTROUTING \\
		nat    & POSTROUTING \\ \hline
	\end{tabular}
\end{table}

\begin{table}[H]
	\caption{Alur paket yang melewati network interface ke proses lokal (input)}
	\label{table:network_to_local}
	\centering
	\begin{tabular}{ll}
		\hline
		\rowcolor[HTML]{C0C0C0} 
		table  & chain      \\ \hline
		mangle & PREROUTING \\
		nat    & PREROUTING \\
		mangle & INPUT      \\
		filter & INPUT      \\ \hline
	\end{tabular}	
\end{table}

\begin{table}[H]
	\caption{Alur paket yang datang dari proses lokal ke network interface (output)}
	\label{table:local_to_network}
	\centering
	\begin{tabular}{ll}
		\hline
		\rowcolor[HTML]{C0C0C0} 
		table  & chain       \\ \hline
		mangle & OUTPUT      \\
		nat    & OUTPUT      \\
		filter & OUTPUT      \\
		mangle & POSTROUTING \\
		nat    & POSTROUTING \\ \hline
	\end{tabular}
\end{table}

\begin{table}[H]
	\caption{Alur paket yang datang dari proses lokal ke proses lokal lainnya (local)}
	\label{table:local_to_local}
	\centering
	\begin{tabular}{ll}
		\hline
		\rowcolor[HTML]{C0C0C0} 
		table  & chain  \\ \hline
		mangle & OUTPUT \\
		nat    & OUTPUT \\
		filter & OUTPUT \\
		filter & INPUT  \\
		mangle & INPUT  \\ \hline
	\end{tabular}
\end{table}

\chapter{Analisis dan Perancangan}

\section{Analisis Malware WannaCry}

Pada bagian ini, peneliti memfokuskan analisa malware WannaCry pada analisa lalu lintas paket yang dilakukan oleh host terinfeksi. Analisis malware WannaCry dilakukan dengan melakukan \textit{sniffing} dengan menggunakan 3 host dengan jaringan terisolasi seperti pada gambar \ref{fig:analisis_malware_net}. Ketiga host tersebut memiliki arsitektur yang sama yakni x86\_64 sehingga tidak ada alignment (seperti little endian dan big endian) berbeda.

\begin{figure}[H]
	\centering
	\includegraphics[width=400px]{resources/analisis_malware_net.png}
	\caption{Susunan jaringan untuk analisis malware WannaCry}
	\label{fig:analisis_malware_net}
\end{figure}

Host linux (192.168.1.1) menjadi host dengan dua interface yang dijadikan \textit{bridge}. Sehingga host Windows 7 terinfeksi (192.168.1.100) dapat berkomunikasi dengan host Windows 7 SP1 (192.168.1.101) hanya melalui 192.168.1.1. Kemudian pada host linux dilakukan sniffing dengan menggunakan tcpdump. dengan command sebagai berikut:

\begin{verbatim}
$ tcpdump -s0 -i br0 -vv -w output.wannacry-1.pcap
\end{verbatim}

\section{Karakteristik Malware WannaCry}

Dari pengelompokan malware menjadi worm, virus dan trojan horse menurut (\cite{idika2007survey}) maka WannaCry digolongkan sebagai worm. Sesuai dengan karakteristik yang disebutkan WannaCry memiliki kapabilitas menginfeksi melalui network. WannaCry memiliki dua buah bagian utama: ransomware, dan worm penginfeksi, yang melakukan penyebaran melalui protokol SMB. Jika sebuah host hanya menjalankan bagian ransomware saja, maka tidak akan ada penyebaran yang dilakukan, seperti ditunjukan pada Gambar \ref{fig:no_infect_action}. Sedangkan jika host menjalankan bagian dropper, dropper tersebut akan menjalankan worm penginfeksi dan sekaligus menjalankan ransomware. Pada Gambar \ref{fig:infect_action} terlihat host 192.168.1.100  mencoba melakukan koneksi ke port \verb|445/tcp| setiap host yang berada pada subnet yang sama.

\begin{figure}[H]
	\centering
	\includegraphics[width=\textwidth]{resources/no_infect_action.png}
	\caption{Paket pada host terinfeksi ransomware tanpa worm penginfeksi}
	\label{fig:no_infect_action}
\end{figure}

\begin{figure}[H]
	\centering
	\includegraphics[width=\textwidth]{resources/infect_action.png}
	\caption{Paket pada host terinfeksi ransomware dengan dropper}
	\label{fig:infect_action}
\end{figure}

Ransomware merupakan kategori malicious software yang ketika dijalankan akan menonaktifkan fungsi tertentu dari komputer dengan sebuah cara. Kemudian ransomware akan menampilkan pesan untuk meminta pembayaran untuk mengembalikan fungsi yang dinonatifkan. Sehingga malware seakan-akan melakukan penyandraan terhadap komputer. (\cite{o2012ransomware}).

Dari riset yang telah dilakukan oleh \cite{islam2018smb}, WannaCry melakukan exploit terhadap vulnerability yang menurut EternalBlue dan DoublePulsar yang ada pada implementasi SMB1.

\subsection{Vulnerability EternalBlue}
EternalBlue merupakan vulnerability yang diakibatkan oleh 3 buah bug (\cite{islam2018smb}) dan (\cite{grossman2017check}) yakni:
\begin{enumerate}
	\item Wrong casting bug
	\item Wrong parsing function bug
	\item Non-paged pool allocation bug
\end{enumerate}

Pada host 192.168.1.100 paket yang dikirimkan malware untuk mengeksploitasi vulnerability seperti yang disebutkan (\cite{islam2018smb}) terlihat pada Gambar \ref{fig:trans_nop}.

\begin{figure}[H]
	\centering
	\includegraphics[width=\textwidth]{resources/trans_nop.png}
	\caption{Paket pada host terinfeksi ransomware dengan dropper}
	\label{fig:trans_nop}
\end{figure}


\subsection{Vulnerability DoublePulsar}

\section{Analisis Payload SMB oleh WannaCry}

Pada malware wannacry

\section{Deteksi Protokol oleh nDPI}

\section{Deteksi signature dengan string matching}


\section{Penangkalan paket \textit{malicious} dengan firewall}


\section{Penentuan rule iptables untuk melakukan \textit{block} pada \textit{payload} WannaCry}




\chapter{Implementasi dan Pengujian}

Bab ini mecakup implementasi dan pengujian sistem hasil rancangan yang telah dijelaskan pada bab sebelumnya. Pada bab ini bagian implementasi tidak mencakup seluruh proses dan detail pada sistem. Detail tersebut dapat dilihat pada hasil implementasi (https://github.com/ibrohimislam/ngfilter). Sedangkan pada bab ini dijelaskan bagian menarik dari implementasi tersebut.

Pada bagian pengujian dijelaskan analisis pengujian yang memungkinkan dan alasan memilih pengujian. Pengujian yang dilakukan dengan transparent-firewall dipilih dengan alasan feasibilitas waktu yang dijelaskan pada bab ini. Kemudian dilanjutkan dengan hasil pengujian dan pembahasan.

\section{Arsitektur Implementasi Sistem}
Perancangan dilakukan dengan menggunakan framework NetFilter. NetFilter merupakan framework \textit{native} yang dimiliki oleh Linux untuk melakukan pemrosesan paket. Framework ini memberikan kemampuan untuk dapat menambahkan sebuah filter kepada user dengan berinteraksi dengan kakas pada \textit{userspace} yakni iptables.

\begin{figure}[H]
	\centering
	\includegraphics[width=230px]{resources/ngfilter_architecture.png}
	\caption{Arsitektur Pendekatan Dynamic}
	\label{fig:ngfilter_architecture}
\end{figure}

Secara garis besar arsitektur implementasi dapat dilihat pada diagram \ref{fig:ngfilter_architecture}. \verb|xt_ngfilter| dan \verb|libxt_ngfilter| merupakan bagian yang menjadi kontribusi implementasi dari tugas akhir ini. Dalam pemetaannya dengan arsitektur logicalnya, \verb|xt_ngfilter| merupakan implementasi komponen \textit{data interpreter} yang memiliki kemampuan DPI. Sedangkan \verb|libxt_ngfilter| merupakan implementasi komponen \textit{rule interpreter}.

Arsitektur implementasi sistem ini hanya memiliki 4 komponen, yakni: Netfilter, iptables,  \verb|xt_ngfilter|, dan \verb|libxt_ngfilter|. Komponen Netfilter pada sistem ini merupakan implementasi komponen \textit{data gatherer}, \textit{matcher}, \textit{filter}. Netfilter pada Linux memang diimplementasi untuk melakukan hal ini.

Bagian utama yang menjadi kontribusi adalah \verb|xt_ngfilter| dan \verb|libxt_ngfilter|. Bentuk diagram kelas nya dapat dilihat pada gambar \ref{fig:class_diagram}.

\begin{figure}[H]
	\centering
	\includegraphics[width=370px]{resources/ngfilter_class_diagram.png}
	\caption{Diagram Kelas}
	\label{fig:class_diagram}
\end{figure}

\verb|xtables_match| merupakan sebuah struct yang memiliki properti \verb|help|, \verb|init|, \verb|parse|, \verb|final_check|, \verb|print|, \verb|save|, \verb|extra_opts| yang merupakan method yang akan diinvoke ketika melakukan manipulasi terhadap rule iptables.

\verb|xt_match| merupakan sebuah struct yang memiliki properti \verb|match|, \verb|checkentry|, dan \verb|destroy| yang akan diinvoke pada kernel space ketika sebuah paket melewati hook dan kemudian melakukan match terhadap modul yang diimplementasi. Fungsi \verb|match| merupakan bagian yang menentukan apakah sebauh packet match dengan parameter-parameter yang ada pada rule.


\section{Traceability Implementasi}

Pada tabel \ref{table:system_traceability_implementation} berikut ditunjukan kerunutan desain dan kebutuhan. Penangkalan paket (SR1) dapat menggunakan \textit{action} \verb|DROP| dari iptables. SR2, State machine dapat didipenuhi dengan menggunakan modul \verb|CONNTRACK| seperti dijelaskan pada subbab sebelumnya. SR3, deteksi protokol level aplikasi dapat dipenuhi dengan menggunakan modul nDPI-Netfilter. SR4, deteksi signature yang paham terhadap struktur data diimplementasi dalam bentuk modul kernel \verb|xt_ngfilter|. Kemudian untuk memenuhi SR5, menambahkan rule menggunakan kakas iptables dengan membuat modul \textit{userspace} untuk berinteraksi dengan modul kernel \verb|xt_ngfilter|.

\begin{table}[H]
	\caption{Traceability Implementasi Sistem}
	\label{table:system_traceability_implementation}
	\begin{tabularx}{\textwidth}{|l|X|X|}
		\hline
		\textbf{No} & \textbf{Kebutuhan} & \textbf{Desain} \\ \hline
		SR1 & Penangkalan paket & menggunakan action DROP oleh iptables\\ \hline 
		SR2 & State machine &  menggunakan modul mangle \verb|CONNTRACK| pada rule iptables \\ \hline
		SR3 & Deteksi protokol level aplikasi & menggunakan modul nDPI-NetFilter\\ \hline
		SR4 & Deteksi signature yang paham terhadap struktur data& class \verb|xt_ngfilter| pada diagram kelas \ref{fig:class_diagram} \\ \hline
		SR5 & Menambahkan rule & class \verb|libxt_ngfilter| pada diagram kelas \ref{fig:class_diagram} \\ \hline
	\end{tabularx}
\end{table}


\section{Implementasi ngfilter}

Implementasi dilakukan dengan membuat sebuah modul kernel \verb|xt_ngfilter| dan  \textit{shared library} \verb|libxt_ngfilter.so|.
Modul kernel digunakan untuk melakukan pencocokan, sedangkan \verb|libxt_ngfilter.so| berinteraksi dengan user melalui iptables.

\begin{lstlisting}
static struct xtables_match ngfilter_mt_reg = {
	.version = XTABLES_VERSION,
	.name = "ngfilter",
	.revision = 0,
	.family = NFPROTO_IPV4,
	.size = XT_ALIGN(sizeof(struct xt_ngfilter_mtinfo)),
	.userspacesize = XT_ALIGN(sizeof(struct xt_ngfilter_mtinfo)),
	.help = ngfilter_match_help,
	.init = ngfilter_match_init,
	.parse = ngfilter_match_parse,
	.final_check = ngfilter_match_check,
	.print = ngfilter_match_print,
	.save = ngfilter_match_save,
	.extra_opts = ngfilter_match_opts,
};
\end{lstlisting}

\begin{lstlisting}
static struct xt_match ngfilter_match4_reg __read_mostly = {
	.name = "ngfilter",
	.revision = 0,
	.family = NFPROTO_IPV4,
	.match = ngfilter_match,
	.checkentry = ngfilter_match_check,
	.destroy = ngfilter_match_destroy,
	.matchsize = sizeof(struct xt_ngfilter_mtinfo),
	.me = THIS_MODULE,
};
\end{lstlisting}

Instance struct \verb|xtables_match| digunakan untuk mendefinisikan modul match yang diimplementasi di userspace.
Property \verb|help|, \verb|init parse|, \verb|final_check|, \verb|print|, \verb|save|, dan \verb|extra_opts| merupakan \textit{function pointer} yang mengarah ke fungsi yang telah diimplementasi.

\begin{itemize}
\item \verb|ngfilter_match_init| dieksekusi saat modul di-\textit{register}.
\item \verb|ngfilter_match_exit| dieksekusi saat modul di-\textit{unregister}.
\item \verb|ngfilter_match_help| digunakan untuk menampilkan pesan bantuan ketika dipanggil dari \verb|iptables|.
\item \verb|ngfilter_match_parse| dieksekusi saat perintah dengan modul match \verb|ngfilter| ditambahkan ke rule. Fungsi ini melakukan mapping dari parameter perintah iptables ke dalam struktur data \verb|xt_ngfilter_mtinfo|.
\item \verb|ngfilter_match_check| merupakan fungsi yang dieksekusi saat melakukan validasi rule dengan menggunakan modul match \verb|ngfilter|.
\item \verb|ngfilter_match_print| merupakan fungsi untuk menampilkan rule yang sedang aktif. Fungsi ini dieksekusi ketika perintah \verb|iptables -L| dijalankan.
\item \verb|ngfilter_match_save| merupakan fungsi yang digunakan ketika perintah \verb|iptables-save| dijalankan. Fungsi ini melakukan mapping dari struktur data ke parameter perintah iptables sehingga dapat disimpan.
\end{itemize}

Berikut adalah definisi struct yang digunakan untuk berkomunikasi antara user-space dan kernel module.

\begin{lstlisting}
#define MAX_PATTERN_LENGTH 256
struct xt_ngfilter_mtinfo {
	unsigned char pattern[MAX_PATTERN_LENGTH];
	unsigned char smb_command;
	__u8 flags;
};
\end{lstlisting}

\section{Perancangan State Machine dengan Iptables}

Pada bagian ini dibahas bagaimana iptables dapat menjalankan state machine pada satu koneksi TCP. Hal ini digunakan pada perancangan selanjutnya untuk menjalankan \textit{rule} \textit{stateful} untuk mendapatkan \textit{signature} dari WannaCry. State machine dijalankan pada satu koneksi karena pada analisis ditemukan bahwa pada kasus WannaCry solusi ini cukup.

Kebutuhan yang diperlukan dalam melakukan implementasi state machine adalah sebagai berikut. state machine dapat menyimpan state; state machine dapat melakukan transisi state berdasarkan \textit{input} dan state saat ini; dan state machine dapat menentukan apakah sebuah \textit{input} dapat diterima atau tidak.

Implementasi penyimpanan state, dalam konteks ini state dalam satu koneksi TCP, dapat dilakukan denganmodul \verb|CONNTRACK| pada iptables yang dapat menyimpan state dalam bentuk 32-bit. Modul \verb|CONNTRACK| dapat digunakan pada tabel \verb|MANGLE|. Berikut adalah perintah yang dapat digunakan.

\begin{lstlisting}
-t mangle -A PREROUTING -j CONNMARK --restore-mark
-t mangle -A POSTROUTING -j CONNMARK --save-mark
\end{lstlisting} 

Perintah di atas dapat menyimpan tanda yang diatur dengan modul \verb|MARK|, dan mengembalikannya tanda pada paket selanjutnya. Akibatnya seluruh tanda yang ada pada sebuah paket akan menjadi state koneksi ketika masuk ke chain \verb|POSTROUTING|. Kemudian seluruh paket yang masuk setelah koneksi memiliki state, akan ditandai ketika masuk ke chain \verb|PREROUTING|.

Implementasi transisi state dapat dilakukan dengan modul \verb|MARK|. Hal ini dapat dilakukan dengan cara melakukan pengecekan terhadap tanda pada paket dan mengubah tanda jika menemukan kondisi yang cocok. Berikut adalah perintah yang dapat digunakan untuk melakukan transisi state.

\begin{lstlisting}
-m mark --mark $CURRENT_STATE $ALPHABET -j MARK --set-xmark $NEW_STATE
\end{lstlisting}

Perintah di atas akan melakukan pengecekan bahwa sebuah paket berada dalam sebuah state \verb|$CURRENT_STATE| dan menerima \textit{input} berupa \verb|$ALPHABET|. Kemudian ketika kondisi tersebut dipenuhi, maka tanda akan diubah ke \verb|$NEW_STATE|. Sehingga perintah tersebut dapat merepresentasikan fungsi transisi state.

Implementasi state machine dapat menerima sebuah \textit{input} atau tidak dapat dilakukan dengan menggunakan perintah \verb|-j $ACTION|. Perintah itu harus dipasangkan setelah melakukan pengecekan terhadap state final. Bentuk lengkap perintahnya adalah sebagai berikut.

\begin{lstlisting}
-m mark --mark $FINAL_STATE -j $ACTION
\end{lstlisting}

\section{Implementasi Rule iptables}

Penangkalan paket \textit{malicious} dapat dilakukan dengan menggunakan rule iptables dengan menambahkan modul yang telah diimplementasi sesuai desain pada subbab III.6. Penangkalan paket \textit{malicious} dapat ditangani dengan menggunakan modul \verb|ndpi-netfilter| dan modul \verb|ngfilter| dengan rule iptables sebagai berikut:

\begin{lstlisting}
-t mangle -A PREROUTING -j CONNMARK --restore-mark
-t mangle -A POSTROUTING -j CONNMARK --save-mark
-A FORWARD -m ndpi --smb  -m ngfilter --smb-command a0 -j MARK --set-xmark 0x1/0xffffffff
-A FORWARD -m mark --mark 0x1 -j LOG --log-prefix "MARK 1: "
-A FORWARD -m mark --mark 0x1 -m ngfilter --smb-command 33 -j MARK --set-xmark 0x2/0xffffffff
-A FORWARD -m mark --mark 0x2 -j LOG --log-prefix "MARK 2: "
-A FORWARD -m mark --mark 0x2 -j DROP
\end{lstlisting} 

\section{Analisis Pengujian}

Berkaitan dengan cara penyebaran worm, terdapat 2 penempatan firewall yang perlu diperhatikan, yaitu firewall di antara subnet dan firewall dalam sebuah subnet. Penempatan ini berbeda perilakunya jika dilihat dari sifat worm yang melakukan \textit{local-scanning} dan \textit{random-scanning}. Kedua jenis penempatan dapat mendeteksi random-scanning namun tidak dapat mendeteksi \textit{local-scanning}.

Firewall di antara subnet merupakan firewall yang bekerja sebagai router (layer TCP/IP) sekaligus melakukan filtering. Firewall umumnya ditemukan dengan penempatan ini. Firewall dengan penempatan ini dapat melakukan penjagaan lalu lintas antar subnet. Namun tidak dapat menjamin pengamanan antar host dalam subnet tersebut. Sehingga, jika terdapat host yang terinfeksi worm dan melakukan \textit{subnet-local-scanning} firewall tidak dapat mendeteksi aktivitas tersebut.

Cara penempatan firewall pada sebuah subnet dapat melakukan penjagaan pada sekelompok host dan memisahkannya dari host yang lain dalam sebuah subnet. Firewall dengan penempatan ini umumnya disebut \textit{transparent firewall}. Penempatan ini dapat mendeteksi aktivitas \textit{subnet-local-scanning} jika melintasi firewall.

Karena dibatasi waktu, pengujian dengan menempatkan firewall di antara subnet tidak dapat dilakukan karena memerlukan sumber daya dan waktu yang tidak dapat dipenuhi.  Keadaan tersebut dapat diperkirakan seperti persamaan \ref{eqn:prob_one}-\ref{eqn:expected_time}.

Jika seperti pada persamaan \ref{eqn:prob_one}, $P(1)$ adalah kemungkinan kemunculan sebuah ip maka $P(n)$ (persamaan \ref{eqn:prob_n}) merupakan kemungkinan muncul salah satu dari n ip. Sehingga diperlukan ekspektasi percobaan yang diperlukan untuk menemukan ip adalah $E(n)$ kali percobaan (persamaan \ref{eqn:expectation_n}). Jika kecepatan \textit{scanning} adalah $v(1)$ dan kecepatan m buah host adalah $v(m)$, $v(m)$ dapat didekati dengan m kali $v(1)$ (persamaan \ref{eqn:velocity_m}). Jika perkiraan waktu yang dibutuhkan untuk menemukan 1 dari n ip, dengan m host yang melakukan percobaan dengan kecepatan $v(1)$ adalah $t(n,m)$, maka waktu yang dibutuhkan adalah jumlah percobaan yang diperlukan dibagi dengan kecepatan.

Dari pengamatan yang dilakukan kecepatan \textit{scanning} secara acak adalah 1064 ip per menit. Maka dengan persamaan \ref{eqn:expected_time} dengan 32 host terinfeksi dan 32 host target dibutuhkan 3 hari untuk mendapatkan satu serangan. Jika 1 serangan dianggap satu data maka untuk mendapatkan 10 data setidaknya diperlukan waktu 30 hari.

\begin{equation}
\label{eqn:prob_one}
P(1) = \frac{1}{2^{32}-1}
\end{equation}

\begin{equation}
\label{eqn:prob_n}
P(n) = \frac {n}{2^{32}-1}
\end{equation}

\begin{equation}
\label{eqn:expectation_n}
E(n) = \frac{1}{P(n)} = \frac{2^{32}-1}{n}
\end{equation}

\begin{equation}
\label{eqn:velocity_m}
v(m) \approx m \times v(1)
\end{equation}

\begin{equation}
\label{eqn:expected_time}
t(n,m) = \frac{E(n)}{v(m)} \approx \frac{2^{32}-1}{n \times m \times v}
\end{equation}

\section{Skenario Pengujian Akurasi}

Pengujian dilakukan menggunakan membandingkan hasil percobaan kontrol dengan percobaan yang menggunakan firewall hasil implementasi. Percobaan dilakukan pada sebuah \textit{hypervisor} yang memiliki spesifikasi pada tabel \ref{table:hypervisor_specification}. Percobaan ini dilakukan dengan menggunakan 10 virtual machine yang tidak terinfeksi, 1 virtual machine yang telah terinfeksi dan 1 firewall.

\begin{table}[H]
	\caption{Spesifikasi hypervisor yang digunakan untuk melakukan pengujian}
	\label{table:hypervisor_specification}
	\begin{tabularx}{\textwidth}{|l|X|}
		\hline
		\textbf{Spesifikasi} & \textbf{Spesifikasi yang digunakan} \\ \hline
		\textit{Processor} & Intel(R) Xeon(R) CPU E5-2620 0 @ 2.00GHz \\ \hline 
		\textit{Virtualization Infrastructure} & Qemu/KVM \\ \hline
		Linux & Linux 3.10.0-862.11.6.el7.x86\_64 \#1 SMP Tue Aug 14 21:49:04 UTC 2018 x86\_64 x86\_64 x86\_64 GNU/Linux \\ \hline
	\end{tabularx}
\end{table}


Percobaan dilakukan dengan memisahkan sebuah subnet 192.168.1.0/24 menjadi dua bagian,\textit{internal\_network} dan \textit{external\_network} seperti pada gambar \ref{fig:validation_scenario}. Kedua bagian tersebut kemudian dihubungkan dengan sebuah transparent firewall. Kemudian masing-masing bagian ditempatkan 5 host. \textit{external\_network} ditempatkan 5 host untuk menunjukkan bahwa WannaCry dapat menginfeksi host-host yang berada pada jaringan yang sama. 

Percobaan kontrol dan percobaan firewall hasil implementasi akan memperlakukan \textit{internal\_network} dengan berbeda. Jika pada percobaan kontrol, firewall tidak menerapkan \textit{rule} sama sekali pada chain FORWARD. Sedangkan pada percobaan firewall hasil implementasi akan menerapkan \textit{rule} hasil implementasi. Sehingga hasil percobaan dapat menunjukkan bagaimana perilaku malware akibat \textit{rule} yang diterapkan.

Untuk mendapatkan data perilaku host-host baik pada \textit{internal\_network} maupun \textit{external\_network} pada firewall dilakukan \textit{packet capture}. Packet capture tersebut diharapkan dapat menjelaskan keadaan network eksternal dan network internal. Keadaan penting yang perlu dilakukan pengamatan adalah kapan sebuah host terinfeksi. Dengan menggunakan definisi host terinfeksi yang telah dijelaskan sebelumnya, seharusnya dapat dideteksi ketika sebuah host mulai melakukan \textit{local-scanning}.

Seluruh proses pengambilan data menggunakan script otomasi yang dijelaskan pada lampiran A. Masing-masing percobaan dilakukan 30 menit dengan menempatkan sebuah host terinfeksi pada \textit{external\_network} pada detik ke 30. Kemudian setelah 30 menit \textit{packet capture} dihentikan dan seluruh host diatur ulang dengan melakukan \textit{cloning} host yang telah disediakan.

\begin{figure}[H]
	\centering
	\includegraphics[width=180px]{resources/skenario_pengujian.png}
	\caption{Susunan jaringan skenario pengujian}
	\label{fig:validation_scenario}
\end{figure}

\section{Hasil Pengujian Akurasi}

Data dari hasil percobaan berbentuk file pcap (\textit{packet capture}) dilakukan pengolahan untuk mendapatkan perkiraan waktu host terinfeksi. Hasil pengolahan data dapat dilihat di Lampiran B berisi data perkiraan waktu host terinfeksi. Kemudian data tersebut diakumulasikan untuk setiap percobaan untuk mendapatkan data pada suatu waktu telah ada berapa host terinfeksi.

\begin{figure}[H]
	\centering
	\includegraphics[width=\textwidth]{resources/infection_control_over_time.png}
	\caption{Grafik host terinfeksi terhadap waktu (percobaan kontrol)}
	\label{fig:infection_control_over_time}
\end{figure}

Grafik akumulasi host terinfeksi pada percobaan kontrol dapat dilihat pada gambar \ref{fig:infection_control_over_time}. Grafik tersebut berisi 10 data percobaan kontrol. Dari 10 percobaan tersebut dapat diperkirakan setelah 900 detik sudah terdapat lebih dari 5 host terinfeksi baik dari \textit{internal\_network} maupun \textit{external\_network}.

\begin{table}[H]
	\caption{Akumulasi detik ke-1800 host terinfeksi (percobaan kontrol)}
	\label{table:1800s_all_network_control}
	\begin{center}
		\begin{tabularx}{300px}{|X|r|}
			\hline
			\multicolumn{1}{|l}{\textbf{Jumlah host terinfeksi}} & \multicolumn{1}{|l|}{\textbf{Teramati}} \\ \hline
			5 & 1 percobaan\\ \hline
			7 & 1 percobaan\\ \hline
			8 & 3 percobaan\\ \hline
			9 & 1 percobaan\\ \hline
			10 & 4 percobaan\\ \hline
		\end{tabularx}
	\end{center}
\end{table}

\begin{table}[H]
	\caption{Akumulasi detik ke-1800 host \textit{internal\_network} terinfeksi (percobaan kontrol)}
	\label{table:1800s_internal_network_control}
	\begin{center}
		\begin{tabularx}{300px}{|X|r|}
			\hline
			\multicolumn{1}{|l}{\textbf{Jumlah host \textit{internal\_network} terinfeksi}} & \multicolumn{1}{|l|}{\textbf{Teramati}} \\ \hline
			2 & 1 percobaan\\ \hline
			3 & 1 percobaan\\ \hline
			4 & 4 percobaan\\ \hline
			5 & 4 percobaan\\ \hline
		\end{tabularx}
	\end{center}
\end{table}

Pada tabel \ref{table:1800s_all_network_control} dapat dilihat persebaran hasil pengamatan host terinfeksi pada \textit{internal\_network} maupun \textit{external\_network}. Sedangkan pada tabel \ref{table:1800s_internal_network_control} dapat dilihat persebaran pengamatan host terinfeksi pada \textit{internal\_network}. Kedua distribusi ini dapat menggambarkan perilaku infeksi WannaCry.

\begin{table}[H]
	\caption{Akumulasi detik ke-1800 host terinfeksi (percobaan firewall implementasi)}
	\label{table:1800s_all_network_firewalled}
	\begin{center}
		\begin{tabularx}{300px}{|X|r|}
			\hline
			\multicolumn{1}{|l}{\textbf{Jumlah host terinfeksi}} & \multicolumn{1}{|l|}{\textbf{Teramati}} \\ \hline
			1 & 40 percobaan\\ \hline
			2 & 15 percobaan\\ \hline
			3 & 16 percobaan\\ \hline
			4 & 3 percobaan\\ \hline
			5 & 8 percobaan\\ \hline
			4 & 1 percobaan\\ \hline
		\end{tabularx}
	\end{center}
\end{table}

\begin{figure}[H]
	\centering
	\includegraphics[width=\textwidth]{resources/infection_control_over_time_firewalled.png}
	\caption{Grafik host terinfeksi terhadap waktu (percobaan implementasi firewall)}
	\label{fig:infection_control_over_time_firewalled}
\end{figure}

Grafik akumulasi host terinfeksi pada percobaan implementasi firewall dapat dilihat pada gambar \ref{fig:infection_control_over_time_firewalled}. Pada gambar tersebut diamati setelah detik ke-600 tidak terjadi infeksi. Jumlah maksimum infeksi yang terjadi sebanyak 6 host yakni host pada \textit{external\_network}. Persebaran pengamatan host terinfeksi pada keseluruhan network dapat dilihat pada tabel \ref{table:1800s_all_network_firewalled}. Dari 83 percobaan tidak diamati infeksi terjadi pada \textit{internal\_network}.

\section{Skenario Pengukuran Kinerja}

Pengujian performa dilakukan dengan menggunakan kakas \verb|iperf3| dengan versi \verb|3.1.3-1|. Pengujian dapat kinerja dapat dilakukan pada sembarang protokol TCP karena kinerja iptables tidak akan dipengaruhi oleh protokol. Kinerja iptables bergantung pada banyaknya \textit{rule} dan seberapa kompleks algoritma match dari masing-masing modul yang digunakan.

Seperti yang dilakukan sebelumnya, pengukuran dilakukan dengan dua percobaan yakni percobaan kontrol dan percobaan firewall hasil implementasi. Pengujian ini bertujuan untuk menunjukkan seberapa besar penurunan kinerja sistem akibat \textit{rule} yang diterapkan. Sehingga pada percobaan kontrol \textit{rule} tidak diterapkan sedangkan pada percobaan firewall hasil implementasi \textit{rule} diterapkan. Masing-masing percobaan dilakukan selama 100 detik. 

Terdapat dua host yakni 192.168.1.111 dan 192.168.1.112. 192.168.1.111 berada pada \textit{internal\_network} dan 192.168.1.112 berada pada \textit{external\_network} Perintah berikut dijalankan pada host 192.168.1.111:
\begin{lstlisting}
iperf3 -s
\end{lstlisting}

\noindent Kemudian perintah berikut dijalankan pada host 192.168.1.112:
\begin{lstlisting}
iperf3 -c 192.168.1.111 -p 5201 -t 100 --logfile output.txt
\end{lstlisting}


\section{Hasil Pengukuran Kinerja}

Dua percobaan tersebut menghasilkan data bandwidth dan data \textit{packet retry}. Hasil pengukuran \textit{bandwidth} dapat dilihat pada gambar \ref{fig:bandwidth_boxplot}. Hasil pengukuran \textit{packet retry} dapat dilihat pada gambar \ref{fig:retr_boxplot}.

\begin{figure}[H]
	\centering
	\includegraphics[width=250px]{resources/bandwidth_boxplot.jpg}
	\caption{Perbandingan bandwidth}
	\label{fig:bandwidth_boxplot}
\end{figure}

\begin{figure}[H]
	\centering
	\includegraphics[width=250px]{resources/retr_boxplot.jpg}
	\caption{Perbandingan packet retry}
	\label{fig:retr_boxplot}
\end{figure}

Pada data \textit{bandwidth} 30 data hasil pengukuran untuk masing-masing percobaan, menunjukkan penurunan rata-rata bandwidth sebesar 4\%. Sedangkan pada rata-rata \textit{packet retry} diamati kenaikan lebih besar dari 8202\%. Hal ini menunjukkan memang ada penurunan kinerja akibat diterapkannya firewall ini.

\section{Pembahasan}

Pada subbab ini dijelaskan bagaimana tujuan pada bab I berhasil dilakukan berdasarkan data yang didapatkan pada percobaan. Selain itu dalam subbab ini dijelaskan kelemahan dan kemungkinan pengembangan selanjutnya. Kelemahan dan kemungkinan pengembangan selanjutnya yang kemudian disimpulkan dalam bab selanjutnya.

Percobaan kontrol menunjukkan bahwa malware WannaCry tetap dapat menginfeksi meskipun melalui sebuah \textit{bridge}, dalam hal ini juga berlaku sebagai \textit{transparent firewall} namun belum diterapkan \textit{rule}. Percobaan ini menunjukkan pada kondisi tanpa perlakuan, WannaCry dapat menginfeksi \textit{internal\_network}. Jika implementasi dapat mencapai mencegah penyebaran WannaCry maka dengan perlakuan tersebut seharusnya WannaCry tidak dapat menginfeksi \textit{internal\_network}.

Kemudian hasil percobaan firewall implementasi dari 83 data yang diamati, tidak terjadi infeksi pada \textit{internal\_network}. Data tersebut membuktikan bahwa implementasi firewall dapat mencegah infeksi WannaCry dengan signifikan. Meskipun dari 83 data masih belum diamati \textit{false-negative}. \textit{False-negative} seharusnya dapat diamati ketika terjadi infeksi pada \textit{internal\_network}.

Selain itu \textit{false-positive} juga sulit diamati dari data tersebut karena pengujian hanya memperhatikan bagaimana infeksi terjadi. Sedangkan tidak melakukan pengecekan pada protokol SMB pada keadaan normal. Walaupun penulis telah melakukan pengujian pada saat implementasi, dan mengamati bahwa rule firewall tersebut tidak mengganggu proses normal.

Bagian menarik dari data pada Lampiran B, malware ini tidak menginfeksi \textit{external\_network} secara maksimal ketika rule diterapkan. Dari hal ini dapat ditarik sebuah kesimpulan ketika malware ditangani pada suatu sisi, sisi lain dari \textit{network} pun menjadi lebih aman. Hal ini terjadi karena \textit{internal\_network} tidak menjadi host yang terinfeksi dan kemudian ikut menyerang host lain.

Implementasi dalam bentuk \textit{signature-based} memiliki kelebihan untuk dapat menangkap signature dengan tingkat \textit{false-negative} yang kecil. Implementasi pada iptables merupakan bentuk implementasi \textit{realtime dynamic signature-based detection}. Implementasi DPI dalam bentuk ini perlu memperhatikan fragmentasi paket, sehingga paket perlu ditunggu hingga lengkap untuk dapat dianalisis. Pada implementasi saat ini masih belum secara detail diperhatikan, sehingga memungkinkan untuk menimbulkan \textit{false-negative}.

Pemilihan \textit{signature} dengan menggunakan \textit{vulnerability} yang di-\textit{exploit} malware dapat menimbulkan \textit{false-positive}. \textit{False-positive} ini terjadi ketika serangan dari malware lain atau attacker yang menggunakan vulnerability yang digunakan pada deteksi. Sehingga hal ini tidak dapat mendeteksi secara presisi serangan yang dideteksi merupakan serangan oleh WannaCry.

Pada pengembangan selanjutnya, seharusnya implementasi dalam bentuk ini dapat digunakan untuk mendeteksi signature malware-malware yang telah diketahui. Hal ini seharusnya dapat menekan \textit{long-tail infection} yang terjadi seperti pada malware WannaCry ini. Namun, tidak semua malware dapat dideteksi dengan implementasi ini. Hal ini karena tidak semua signature infeksi dapat ditemukan dalam satu koneksi seperti yang terjadi pada WannaCry. Oleh karena itu diperlukan implementasi state-machine yang tidak hanya \textit{single-connection oriented}.

Hasil pengukuran kinerja menujukan perbedaan yang dapat diterima dan penurunan tidak signifikan pada bandwidth. Sedangkan pada \textit{packet retry}, terlihat perbedaan yang signifikan antara percobaan kontrol dan percobaan firewall implementasi. Hal ini terjadi karena firewall implementasi memang memerlukan pemrosesan tambahan ketika menerima paket. Penurunan performa ini dalam kadar yang dapat diterima.

\section{Permasalahan Teknis}

Dalam subbab ini dijelaskan hal-hal apa saja yang menjadi beberapa hambatan dalam melakukan implementasi DPI pada firewall. Beberapa masalah mengenai dokumentasi dan permasalahan library yang digunakan, hingga pada proses pengujian.

Hambatan paling dirasakan adalah kurangnya dokumentasi mengenai perubahan yang harus dilakukan untuk modul kernel untuk berubah ke versi yang lebih tinggi. Salah satu akibatnya adalah meski pun tersedia panduan pembuatan modul kernel untuk Netfilter namun karena perbedaan versi tidak dapat dilakukan dengan mudah.

Selain itu implementasi nDPI saat ini merupakan implementasi pada userspace sehingga tidak dapat dengan mudah digunakan pada modul kernel. Pada versi terdahulu nya nDPI memang dapat digunakan pada kernel. Namun pada versi terbaru, sudah tidak men-\textit{support} penggunakan pada kernel. Sehingga diperlukan beberapa patch untuk mengubah library-library userspace yang digunakan nDPI.

Pengujian dengan menggunakan transparent firewall di satu sisi dapat mempermudah dalam mendapatkan data. Namun, pada sisi teknisnya karena transparent firewall dalam implementasinya berupa bridge yang beroperasi pada layer 2. Sedangkan iptables beroperasi pada layer 3. Diperlukan pengaturan khusus pada kernel sehingga forwarding pada layer 2 dapat mengaktifkan hook pada iptables.

\chapter{Penutup}

Pada bab ini disampaikan kesimpulan yang didapatkan dari tugas akhir ini. Kemudian bab ini dilanjutkan dengan saran dari penulis.

\section{Kesimpulan}

Implementasi DPI pada firewall open-source iptables dapat digunakan sebagai penangkal malware yang pada tugas akhir ini dibatasi pada malware WannaCry. Implementasi dilakukan dengan menggunakan teknik \textit{dynamic signature-based}. Pengujian yang dilakukan belum ditemukan \textit{false-negative} pada implementasi DPI ini.

\section{Saran}

Saran-saran untuk pengembangan selanjutnya berdasarkan pengembangan yang telah dilakukan.
\begin{enumerate}
	\item Membuat \textit{Domain Specific Language} untuk merepresentasikan sebuah protokol. Dengan DSL diharapkan penambahan jenis protokol untuk DPI dapat dilakukan dengan lebih mudah.
	\item Implementasi \textit{state machine} yang tidak hanya bergantung pada satu koneksi. Hal ini dapat meningkatkan kemampuan firewall untuk mendeteksi malware yang melakukan infeksi tidak hanya menggunakan \textit{single-connection}.
	\item Implementasi \textit{Deep Packet Inspection} yang memperhatikan performa.
\end{enumerate}


\printbibliography[heading=bibintoc,title={Daftar Referensi}]

\appendix
\part*{LAMPIRAN}

\captionsetup[figure]{list=no}
\captionsetup[table]{list=no}

\chapter{Instrumen Pengujian}

Lampiran ini berisi instrumen pengujian yang digunakan untuk melakukan pengumpulan data. Pengumpulan data dilakukan dengan otomasi pada \textit{hypervisor} yang menggunakan \verb|qemu|. Otomasi dilakukan untuk mengumpulkan data dalam bentuk \verb|pcap|. Lampiran ini ditambahkan untuk mempermudah pembaca untuk mendapatkan data seperti yang dilakukan pada penelitian ini.

\section{start.sh}

Script ini digunakan untuk melakukan otomasi untuk menghidupkan \textit{instance virtual machine}, mengirimkan perintah capture, menghentikan \textit{instance virtual machine} dan mengatur kembali keadaan \textit{instance virtual machine}. Menghidupkan dan menghentikan instance virtual machine dilakukan dengan perintah \verb|virsh|. Mengirimkan perintah capture dengan melakukan \verb|SSH| ke firewall. Bagian mengatur kembali keadaan dengan menghapus image, dan menyalin dari master.

\begin{lstlisting}[language=Bash]
#!/bin/bash
echo "starting..."
for i in `seq 1 10`; do virsh start ibrohim-winsrv-$i; done;
echo "done."

sleep 5m;
ssh root@167.205.3.202 "screen -dm bash capture.sh"
sleep 1m;

virsh start ibrohim-wannacry-1
sleep 30m;
virsh destroy ibrohim-wannacry-1

echo "stopping..."
for i in `seq 1 10`; do virsh destroy ibrohim-winsrv-$i; done;
echo "done."

echo "cleanup..."
for i in `seq 1 10`; do rm -vf ibrohim-winsrv-$i.img; done;
rm -vf ibrohim-wannacry.img
echo "done."

echo "cloning..."
for i in `seq 1 10`; do cp -v ibrohim-winsrv-master.img ibrohim-winsrv-$i.img; done;
cp -v ibrohim-wannacry-master.img ibrohim-wannacry.img
echo "done."
\end{lstlisting}

\section{capture.sh}

Script ini digunakan untuk melakukan \textit{packet capture}. \textit{Packet capture} dilakukan dengan menggunakan perintah \verb|tcpdump|.

\begin{lstlisting}[language=Bash]
#!/bin/bash
i="1"
while [ -f output.$i.pcap ]; do let "i++"; done;

file=output.$i.pcap

echo "output: $file"
echo "start capturing..."
tcpdump -s0 -vv -w $file &
sleep 30m; kill $!
echo "done."
\end{lstlisting}


\chapter{Data Percobaan}

\begin{table}[H]
	\caption{tabel data percobaan 1}
	\label{table:data percobaan 1}
	\centering
	\begin{tabular}{|l|l|l|l|l|}
		\hline
		\multirow{2}{*}{\footnotesize{Mac Address}} & \multicolumn{4}{c|}{\footnotesize{Waktu infeksi terdeteksi}} \\ \cline{2-5} 
		& \footnotesize{percobaan 1} & \footnotesize{percobaan 2} & \footnotesize{percobaan 3} & \footnotesize{percobaan 4}\\ \hline
		\footnotesize{\verb|52:54:00:8a:d6:60|} &1 &2 &3 &4 \\ \hline
		\footnotesize{\verb|52:54:00:05:0f:b0|} &1 &2 &3 &4 \\ \hline
		\footnotesize{\verb|52:54:00:53:55:f8|} &1 &2 &3 &4 \\ \hline
		\footnotesize{\verb|52:54:00:94:47:8f|} &1 &2 &3 &4 \\ \hline
		\footnotesize{\verb|52:54:00:4a:9b:54|} &1 &2 &3 &4 \\ \hline
		\footnotesize{\verb|52:54:00:00:e4:8a|} &1 &2 &3 &4 \\ \hline
		\footnotesize{\verb|52:54:00:e4:6a:37|} &1 &2 &3 &4 \\ \hline
		\footnotesize{\verb|52:54:00:5b:e8:4e|} &1 &2 &3 &4 \\ \hline
		\footnotesize{\verb|52:54:00:0e:a5:bb|} &1 &2 &3 &4 \\ \hline
		\footnotesize{\verb|52:54:00:9e:a7:84|} &1 &2 &3 &4 \\ \hline
		\footnotesize{\verb|52:54:00:e2:b8:bf|} &1 &2 &3 &4 \\ \hline
	\end{tabular}
\end{table}


\end{document}
