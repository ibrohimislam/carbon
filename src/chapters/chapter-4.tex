\chapter{Implementasi dan Pengujian}

\section{Implementasi ngfilter}

Implementasi dilakukan dengan membuat sebuah modul kernel \verb|xt_ngfilter| dan  \textit{shared library} \verb|libxt_ngfilter.so|.
Modul kernel digunakan untuk melakukan pencocokan, sedangkan \verb|libxt_ngfilter.so| berinteraksi dengan user melalui iptables.

\begin{lstlisting}
static struct xtables_match ngfilter_mt_reg = {
.version = XTABLES_VERSION,
.name = "ngfilter",
.revision = 0,
.family = NFPROTO_IPV4,
.size = XT_ALIGN(sizeof(struct xt_ngfilter_mtinfo)),
.userspacesize = XT_ALIGN(sizeof(struct xt_ngfilter_mtinfo)),
.help = ngfilter_match_help,
.init = ngfilter_match_init,
.parse = ngfilter_match_parse,
.final_check = ngfilter_match_check,
.print = ngfilter_match_print,
.save = ngfilter_match_save,
.extra_opts = ngfilter_match_opts,
};
\end{lstlisting}

\begin{lstlisting}
static struct xt_match ngfilter_match4_reg __read_mostly = {
	.name = "ngfilter",
	.revision = 0,
	.family = NFPROTO_IPV4,
	.match = ngfilter_match,
	.checkentry = ngfilter_match_check,
	.destroy = ngfilter_match_destroy,
	.matchsize = sizeof(struct xt_ngfilter_mtinfo),
	.me = THIS_MODULE,
};
\end{lstlisting}

Instance struct \verb|xtables_match| digunakan untuk mendefinisikan modul match yang diimplementasi di userspace.
Property \verb|help|, \verb|init parse|, \verb|final_check|, \verb|print|, \verb|save|, dan \verb|extra_opts| merupakan \textit{function pointer} yang mengarah ke fungsi yang telah diimplementasi.

\begin{itemize}
\item \verb|ngfilter_match_init| dieksekusi saat modul di-\textit{register}.
\item \verb|ngfilter_match_exit| dieksekusi saat modul di-\textit{unregister}.
\item \verb|ngfilter_match_help| digunakan untuk menampilkan pesan bantuan ketika dipanggil dari \verb|iptables|.
\item \verb|ngfilter_match_parse| dieksekusi saat perintah dengan modul match \verb|ngfilter| ditambahkan ke rule. Fungsi ini melakukan mapping dari parameter perintah iptables ke dalam struktur data \verb|xt_ngfilter_mtinfo|.
\item \verb|ngfilter_match_check| merupakan fungsi yang dieksekusi saat melakukan validasi rule dengan menggunakan modul match \verb|ngfilter|.
\item \verb|ngfilter_match_print| merupakan fungsi untuk menampilkan rule yang sedang aktif. Fungsi ini dieksekusi ketika perintah \verb|iptables -L| dijalankan.
\item \verb|ngfilter_match_save| merupakan fungsi yang digunakan ketika perintah \verb|iptables-save| dijalankan. Fungsi ini melakukan mapping dari struktur data ke parameter perintah iptables sehingga dapat disimpan.
\end{itemize}

Berikut adalah definisi struct yang digunakan untuk berkomunikasi antara user-space dan kernel module.

\begin{lstlisting}
#define MAX_PATTERN_LENGTH 256
struct xt_ngfilter_mtinfo {
	unsigned char pattern[MAX_PATTERN_LENGTH];
	unsigned char smb_command;
	__u8 flags;
};
\end{lstlisting}

\section{Implementasi Rule iptables}

Penangkalan paket malicious dapat dilakukan dengan menggunakan rule iptables dengan menambahkan modul yang terlah diimplementasi sesuai desain pada subbab III.6. Penangkalan paket \textit{malicious} dapat ditangani dengan menggunakan modul \verb|ndpi-netfilter| dan modul \verb|ngfilter| dengan rule iptables sebagai berikut:

\begin{verbatim}
-A PREROUTING -j CONNMARK --restore-mark
-A POSTROUTING -j CONNMARK --save-mark
-A FORWARD -m ndpi --smb  -m ngfilter --smb-command 0a -j MARK --set-xmark 0x1/0xffffffff
-A FORWARD -m mark --mark 0x1 -j LOG --log-prefix "MARK 1: "
-A FORWARD -m mark --mark 0x1 -m ngfilter --smb-command 33 -j MARK --set-xmark 0x2/0xffffffff
-A FORWARD -m mark --mark 0x2 -j LOG --log-prefix "MARK 2: "
-A FORWARD -m mark --mark 0x2 -j DROP
\end{verbatim} 

\section{Skenario Pengujian}

Pengujian akan dilakukan dengan melakukan penempatan yang sama dengan \ref{fig:analisis_malware_net} dengan memasangkan modul yang telah diimplementasi pada host 192.168.1.1. Setelah modul dipasangkan, rule seperti pada III.7 diterapkan untuk melakukan penangkalan. Kemudian pada sisi host 192.168.1.101 akan dilakukan pengecekan apakah paket signature sampai ke host ini.
Jika berhasil, kemudian dilakukan pengecekan apakah service file sharing normal masih dapat dilakukan.
Kemudian dilakukan perbandingan kecepatan transfer sebelum dan setelah modul yang diimplementasi dipasang. Hal ini dilakukan mengetahui berapa besar latensi yang timbul akibat pemasangan modul.

\section{Hasil Pengujian}

\section{Pembahasan}