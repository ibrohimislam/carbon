\chapter{Instrumen Pengujian}

Lampiran ini berisi instrumen pengujian yang digunakan untuk melakukan pengumpulan data. Pengumpulan data dilakukan dengan otomasi pada \textit{hypervisor} yang menggunakan \verb|qemu|. Otomasi dilakukan untuk mengumpulkan data dalam bentuk \verb|pcap|. Lampiran ini ditambahkan untuk mempermudah pembaca untuk mendapatkan data seperti yang dilakukan pada penelitian ini.

\section{start.sh}

Script ini digunakan untuk melakukan otomasi untuk menghidupkan \textit{instance virtual machine}, mengirimkan perintah capture, menghentikan \textit{instance virtual machine} dan mengatur kembali keadaan \textit{instance virtual machine}. Menghidupkan dan menghentikan instance virtual machine dilakukan dengan perintah \verb|virsh|. Mengirimkan perintah capture dengan melakukan \verb|SSH| ke firewall. Bagian mengatur kembali keadaan dengan menghapus image, dan menyalin dari master.

\begin{lstlisting}[language=Bash]
#!/bin/bash
echo "starting..."
for i in `seq 1 10`; do virsh start ibrohim-winsrv-$i; done;
echo "done."

sleep 5m;
ssh root@167.205.3.202 "screen -dm bash capture.sh"
sleep 1m;

virsh start ibrohim-wannacry-1
sleep 30m;
virsh destroy ibrohim-wannacry-1

echo "stopping..."
for i in `seq 1 10`; do virsh destroy ibrohim-winsrv-$i; done;
echo "done."

echo "cleanup..."
for i in `seq 1 10`; do rm -vf ibrohim-winsrv-$i.img; done;
rm -vf ibrohim-wannacry.img
echo "done."

echo "cloning..."
for i in `seq 1 10`; do cp -v ibrohim-winsrv-master.img ibrohim-winsrv-$i.img; done;
cp -v ibrohim-wannacry-master.img ibrohim-wannacry.img
echo "done."
\end{lstlisting}

\section{capture.sh}

Script ini digunakan untuk melakukan \textit{packet capture}. \textit{Packet capture} dilakukan dengan menggunakan perintah \verb|tcpdump|.

\begin{lstlisting}[language=Bash]
#!/bin/bash
i="1"
while [ -f output.$i.pcap ]; do let "i++"; done;

file=output.$i.pcap

echo "output: $file"
echo "start capturing..."
tcpdump -s0 -vv -w $file &
sleep 30m; kill $!
echo "done."
\end{lstlisting}

