\chapterappendix{Instrumen Pengujian}

Lampiran ini berisi instrumen pengujian yang digunakan untuk melakukan pengumpulan data. Pengumpulan data dilakukan dengan otomasi pada \textit{hypervisor} yang menggunakan \verb|qemu|. Otomasi dilakukan untuk mengumpulkan data dalam bentuk \verb|pcap|. Lampiran ini ditambahkan untuk mempermudah pembaca untuk mendapatkan data seperti yang dilakukan pada penelitian ini.

\sectionappendix{Konfigurasi}

Dalam instrumen pengujian yang digunakan untuk pengumpulan data terdapat beberapa konfigurasi penting, yakni: bridge, dhcp server dan transparent firewall.

\subsectionappendix{/etc/dhcp/dhcpd.conf}

\begin{lstlisting}
option domain-name "local";
option domain-name-servers dhcp.local;
default-lease-time 600;
max-lease-time 7200;
authoritative;

subnet 192.168.1.0 netmask 255.255.255.0 {
	range dynamic-bootp 192.168.1.100 192.168.1.254;
	option broadcast-address 192.168.1.255;
	option routers 192.168.1.1;
}
\end{lstlisting}

\subsectionappendix{/etc/sysconfig/network-scripts/ifcfg-br0}

\begin{lstlisting}
DEVICE=br0
TYPE=Bridge
IPADDR=192.168.1.1
NETMASK=255.255.255.0
ONBOOT=yes
BOOTPROTO=none
NM_CONTROLLED=no
DELAY=0
\end{lstlisting}

\subsectionappendix{/etc/sysconfig/network-scripts/ifcfg-eth0}

\begin{lstlisting}
TYPE=Ethernet
DEVICE=eth0
BOOTPROTO=none
ONBOOT=yes
NM_CONTROLLED=no
BRIDGE=br0
\end{lstlisting}

\subsectionappendix{/etc/sysconfig/network-scripts/ifcfg-eth1}

\begin{lstlisting}
TYPE=Ethernet
DEVICE=eth1
BOOTPROTO=none
ONBOOT=yes
NM_CONTROLLED=no
BRIDGE=br0
\end{lstlisting}

\subsectionappendix{Konfigurasi Khusus Transparent Firewall}

\begin{lstlisting}
echo 1 > /proc/sys/net/bridge/bridge-nf-call-iptables
\end{lstlisting}

\sectionappendix{Otomasi}

\subsectionappendix{start.sh}

Script ini digunakan untuk melakukan otomasi untuk menghidupkan \textit{instance virtual machine}, mengirimkan perintah capture, menghentikan \textit{instance virtual machine} dan mengatur kembali keadaan \textit{instance virtual machine}. Menghidupkan dan menghentikan instance virtual machine dilakukan dengan perintah \verb|virsh|. Mengirimkan perintah capture dengan melakukan \verb|SSH| ke firewall. Bagian mengatur kembali keadaan dengan menghapus image, dan menyalin dari master.

\begin{lstlisting}[language=Bash]
#!/bin/bash
echo "starting..."
for i in `seq 1 10`; do virsh start ibrohim-winsrv-$i; done;
echo "done."

sleep 5m;
ssh root@167.205.3.202 "screen -dm bash capture.sh"
sleep 1m;

virsh start ibrohim-wannacry-1
sleep 30m;
virsh destroy ibrohim-wannacry-1

echo "stopping..."
for i in `seq 1 10`; do virsh destroy ibrohim-winsrv-$i; done;
echo "done."

echo "cleanup..."
for i in `seq 1 10`; do rm -vf ibrohim-winsrv-$i.img; done;
rm -vf ibrohim-wannacry.img
echo "done."

echo "cloning..."
for i in `seq 1 10`; do cp -v ibrohim-winsrv-master.img ibrohim-winsrv-$i.img; done;
cp -v ibrohim-wannacry-master.img ibrohim-wannacry.img
echo "done."
\end{lstlisting}

\subsectionappendix{capture.sh}

Script ini digunakan untuk melakukan \textit{packet capture}. \textit{Packet capture} dilakukan dengan menggunakan perintah \verb|tcpdump|.

\begin{lstlisting}[language=Bash]
#!/bin/bash
i="1"
while [ -f output.$i.pcap ]; do let "i++"; done;
touch output.$i.pcap

file=output.$i.pcap

echo "output: $file"
echo "start capturing..."
tcpdump -s0 -vv -w $file &
sleep 30m; kill $!

echo "done."
\end{lstlisting}

