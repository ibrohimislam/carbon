\chapter{Penutup}

Pada bab ini disampaikan kesimpulan yang didapatkan dari tugas akhir ini. Kemudian bab ini dilanjutkan dengan saran dari penulis.

\section{Kesimpulan}

Implementasi DPI pada firewall open-source iptables dapat digunakan sebagai penangkal malware yang pada tugas akhir ini dibatasi pada malware WannaCry. Implementasi dilakukan dengan menggunakan teknik \textit{dynamic signature-based}. Pengujian yang dilakukan belum ditemukan \textit{false-negative} pada implementasi DPI ini.

\section{Saran}

Saran-saran untuk pengembangan selanjutnya berdasarkan pengembangan yang telah dilakukan.
\begin{enumerate}
	\item Membuat \textit{Domain Specific Language} untuk merepresentasikan sebuah protokol. Dengan DSL diharapkan penambahan protokol untuk dikenali oleh DPI dapat dilakukan dengan lebih mudah.
	\item Implementasi \textit{state machine} yang tidak hanya bergantung pada satu koneksi. Hal ini dapat meningkatkan kemampuan firewall untuk mendeteksi malware yang melakukan infeksi tidak hanya menggunakan \textit{single-connection}.
	\item Implementasi \textit{Deep Packet Inspection} yang memperhatikan performa.
\end{enumerate}