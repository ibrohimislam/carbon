\chapter{Pendahuluan}

\section{Latar Belakang}

\textit{First-generation} firewall sudah tidak dapat lagi memenuhi kebutuhan keamanan. Firewall ini tidak dapat melakukan pencegahan intrusi pada level protokol yang lebih tinggi. Akibatnya, serangan-serangan dapat dilakukan meskipun suatu enterprise telah mengunakan firewall.

Kekurangan yang dimiliki \textit{first-generation} firewall, menjadikan \textit{enterprise} harus penggunakan sistem lain untuk dapat mendeteksi serangan. Sistem lain seperti \textit{Intrusion Detection System} (IDS) dan \textit{Intrusion Prevention System} (IPS). Hal itu menimbulkan biaya lebih yang harus dikeluarkan oleh \textit{enterprise}.

\textit{Next-generation} firewall menurut (Pescatore, 2009) menjadi solusi untuk menyelesaikan masalah ini. Next-generation firewall merupakan firewall yang memiliki fitur first-generation firewall, namun memiliki fitur-fitur IDS maupun IPS yang terintegrasi, dengan menitikberatkan pada deep packet inspection (DPI).

Menurut sebuah survey dari Alcatel-Lucent strategic white paper (The Case for
Network-based Malware Detection, 2014), solusi pengamanan client-based kurang
efektif untuk mencegah malware. Hampir 81\% terinfeksi malware, meskipun telah
dipasangi oleh anti-virus. Sehingga diperlukan sebuah solusi pengamanan yang
tidak dapat di-disable dari sistem. Salah satu solusinya adalah dengan
menggunakan network-based malware detection.

Malware detection menjadi fitur penting yang diperlukan oleh sebuah sistem pengamanan. Namun, saat ini implementasi open-source firewall, yakni pfSense masih memberikan fitur native untuk melakukan deteksi malware pada \textit{layer} aplikasi.

\section{Rumusan Masalah}
Implementasi open-souce firewall (pfSense, 2017) dan (OPNsense, 2017) saat ini
belum melakukan inspeksi tehadap application layer protocol secara generic.
Padahal intrusi seperti malware pada umumnya sulit dideteksi pada transport-layer,
sehingga masih sulit dilakukan.

\section{Tujuan}
Tujuan dari tugas akhir ini adalah:
\begin{enumerate}
	\item Menentukan teknik yang extensible untuk melakukan pendeteksian malware pada network untuk diimplementasikan pada firewall open-source.
	\item Membangun sistem pendeteksi malware pada implementasi firewall open-source.
\end{enumerate}

\section{Batasan Masalah}
Implementasi Network-Based Malware Detection pada firewall open-source
difokuskan pada hal-hal sebagai berikut:
\begin{enumerate}
	\item Implementasi tidak difokuskan untuk meningkatkan performa firewall
	setelah Network-Based Malware Detection diimplementasi.
	\item Implementasi menggunakan komponen-komponen yang sudah ada dari perangkat lunak open-source dengan kapabilitas deep packet inspection.
	\item Implementasi akan dibatasi untuk melakukan deteksi pada \textit{malware} WannaCry.
\end{enumerate}

\section{Metodologi}
Metodologi yang digunakan pada pengerjaan makalah ini, antara lain:
\begin{enumerate}
	\item Studi literatur. Pada studi literatur, dilakukan pencarian referensi mengenai
	definisi-definisi pada domain next-generation firewall. Referensi juga
	digunakan untuk mendapatkan \textit{state of the art} dari domain ini.
	\item Analisis masalah. Dalam tahapan ini, dilakukan analisis metode apa saja
	yang dapat dilakukan untuk mendeteksi malware dalam seuah network, dan sampai sejauh mana implementasi next-generation firewall dapat
	menangani masalah ini.
	\item Perancangan solusi. Hasil analisis masalah kemudian dilakukan pemilihan
	solusi yang tepat dan cukup generic dalam mendeteksi malware.
	\item Implementasi. Pada tahap ini hasil rancangan solusi kemudian
	diimplementasikan pada \textit{open-source next-generation firewall} yang sudah
	ada.
	\item Pengujian dan analisis hasil. Pada tahap ini hasil implementasi dilakukan
	pengujian dengan menggunakan beberapa kasus uji.
\end{enumerate}
