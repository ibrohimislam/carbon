\chapter{Pendahuluan}

Pada bab ini diberikan latar belakang dan garis besar mengenai tugas akhir. Garis besar yang disajikan berupa rumusan masalah, tujuan dan batasannya. Kemudian bab ini diakhiri dengan metodologi yang digunakan dalam pengerjaan tugas akhir.

\section{Latar Belakang}

Saat ini banyak diterapkan \textit{Intrusion Detection System} (IDS) untuk melakukan deteksi ketika terjadinya retasan. Padahal firewall umumnya sudah diterapkan. Hal ini terjadi karena firewall seperti iptables saat ini tidak dapat mendeteksi retasan yang terjadi pada layer di atas network dan transport. Padahal deteksi retasan oleh malware tidak dapat dilakukan pada layer network dan transport.

Menurut sebuah survey dari \textit{Alcatel-Lucent strategic white paper} \cite{alcatel_lucent_2013}, solusi pengamanan \textit{client-based} kurang efektif untuk mencegah malware. Hampir 81\% terinfeksi malware, meskipun telah dipasangi oleh antivirus. Sehingga diperlukan sebuah solusi pengamanan yang tidak dapat di-\textit{disable} dari sistem. Salah satu solusinya adalah dengan menggunakan network-based malware detection.

Malware detection menjadi fitur penting yang diperlukan oleh sebuah sistem pengamanan. Namun, saat ini implementasi \textit{open-source} firewall, yakni iptables masih belum memberikan fitur \textit{native} untuk melakukan deteksi malware pada \textit{layer} aplikasi.

\section{Rumusan Masalah}

Selama ini \textit{open-source} firewall hanya dapat melakukan pengenalan pada header paket pada layer \textit{network} dan \textit{transport}. Sementara itu malware tidak dapat dikenali hanya dengan mengenali header paket pada layer tersebut. Sehingga pada tugas akhir ini dilakukan pengembangan sistem yang mengenali malware dan dapat menangkalnya dengan mengenali data yang berada di atas layer tersebut.

\section{Tujuan}
Tujuan dari tugas akhir ini adalah membangun sistem penangkal malware yang menerapkan \textit{Deep Packet Inspection} (DPI) pada implementasi firewall \textit{open-source}.

\section{Batasan Masalah}
Implementasi DPI pada firewall \textit{open-source} difokuskan pada hal-hal sebagai berikut:
\begin{enumerate}
	\item Implementasi dilakukan pada firewall \textit{open-source} iptables.
	\item Implementasi tidak difokuskan untuk meningkatkan kinerja firewall setelah \textit{DPI} diimplementasi.
	\item Implementasi menggunakan komponen-komponen yang sudah ada dari pustaka \textit{open-source} dengan kapabilitas DPI.
	\item Implementasi akan dibatasi untuk melakukan deteksi pada \textit{malware} WannaCry.
\end{enumerate}

\section{Metodologi}
Metodologi yang digunakan pada pengerjaan tugas akhir ini, antara lain:
\begin{enumerate}
	\item Studi literatur. Pada studi literatur, dilakukan pencarian referensi mengenai
	definisi-definisi pada domain \textit{next-generation firewall}, dan metode apa saja yang dapat 
	dilakukan untuk mendeteksi serangan yang dilakukan malware. Referensi juga
	digunakan untuk mendapatkan \textit{state of the art} dari domain ini.
	\item Analisis. Dalam tahapan ini dilakukan analisis malware dan analisis gap dari
	kakas yang sudah ada untuk membuat kakas yang dapat mendeteksi dan melakukan pencegahan
	serangan dari sebuah malware melalui jaringan.
	\item Perancangan solusi. Hasil analisis yang telah dilakukan dan gap yang telah diketahui,
	dirancang sebuah kakas yang sesuai dengan kebutuhan yang muncul.
	\item Implementasi. Pada tahap ini hasil rancangan kemudian diimplementasikan.
	\item Pengujian dan analisis hasil. Pada tahap ini hasil implementasi dilakukan
	pengujian dengan menggunakan beberapa kasus uji.
\end{enumerate}
